%**************************************%
%*    Generated from PreTeXt source   *%
%*    on 2020-01-28T10:39:37-08:00    *%
%*                                    *%
%*      https://pretextbook.org       *%
%*                                    *%
%**************************************%
\documentclass[oneside,10pt,]{book}
%% Custom Preamble Entries, early (use latex.preamble.early)
%% Default LaTeX packages
%%   1.  always employed (or nearly so) for some purpose, or
%%   2.  a stylewriter may assume their presence
\usepackage{geometry}
%% Some aspects of the preamble are conditional,
%% the LaTeX engine is one such determinant
\usepackage{ifthen}
%% etoolbox has a variety of modern conveniences
\usepackage{etoolbox}
\usepackage{ifxetex,ifluatex}
%% Raster graphics inclusion
\usepackage{graphicx}
%% Color support, xcolor package
%% Always loaded, for: add/delete text, author tools
%% Here, since tcolorbox loads tikz, and tikz loads xcolor
\PassOptionsToPackage{usenames,dvipsnames,svgnames,table}{xcolor}
\usepackage{xcolor}
%% begin: defined colors, via xcolor package, for styling
%% end: defined colors, via xcolor package, for styling
%% Colored boxes, and much more, though mostly styling
%% skins library provides "enhanced" skin, employing tikzpicture
%% boxes may be configured as "breakable" or "unbreakable"
%% "raster" controls grids of boxes, aka side-by-side
\usepackage{tcolorbox}
\tcbuselibrary{skins}
\tcbuselibrary{breakable}
\tcbuselibrary{raster}
%% We load some "stock" tcolorbox styles that we use a lot
%% Placement here is provisional, there will be some color work also
%% First, black on white, no border, transparent, but no assumption about titles
\tcbset{ bwminimalstyle/.style={size=minimal, boxrule=-0.3pt, frame empty,
colback=white, colbacktitle=white, coltitle=black, opacityfill=0.0} }
%% Second, bold title, run-in to text/paragraph/heading
%% Space afterwards will be controlled by environment,
%% dependent of constructions of the tcb title
\tcbset{ runintitlestyle/.style={fonttitle=\normalfont\bfseries, attach title to upper} }
%% Spacing prior to each exercise, anywhere
\tcbset{ exercisespacingstyle/.style={before skip={1.5ex plus 0.5ex}} }
%% Spacing prior to each block
\tcbset{ blockspacingstyle/.style={before skip={2.0ex plus 0.5ex}} }
%% xparse allows the construction of more robust commands,
%% this is a necessity for isolating styling and behavior
%% The tcolorbox library of the same name loads the base library
\tcbuselibrary{xparse}
%% Hyperref should be here, but likes to be loaded late
%%
%% Inline math delimiters, \(, \), need to be robust
%% 2016-01-31:  latexrelease.sty  supersedes  fixltx2e.sty
%% If  latexrelease.sty  exists, bugfix is in kernel
%% If not, bugfix is in  fixltx2e.sty
%% See:  https://tug.org/TUGboat/tb36-3/tb114ltnews22.pdf
%% and read "Fewer fragile commands" in distribution's  latexchanges.pdf
\IfFileExists{latexrelease.sty}{}{\usepackage{fixltx2e}}
%% Footnote counters and part/chapter counters are manipulated
%% April 2018:  chngcntr  commands now integrated into the kernel,
%% but circa 2018/2019 the package would still try to redefine them,
%% so we need to do the work of loading conditionally for old kernels.
%% From version 1.1a,  chngcntr  should detect defintions made by LaTeX kernel.
\ifdefined\counterwithin
\else
    \usepackage{chngcntr}
\fi
%% Text height identically 9 inches, text width varies on point size
%% See Bringhurst 2.1.1 on measure for recommendations
%% 75 characters per line (count spaces, punctuation) is target
%% which is the upper limit of Bringhurst's recommendations
\geometry{letterpaper,total={340pt,9.0in}}
%% Custom Page Layout Adjustments (use latex.geometry)
%% This LaTeX file may be compiled with pdflatex, xelatex, or lualatex executables
%% LuaTeX is not explicitly supported, but we do accept additions from knowledgeable users
%% The conditional below provides  pdflatex  specific configuration last
%% The following provides engine-specific capabilities
%% Generally, xelatex is necessary for non-Western fonts
\ifthenelse{\boolean{xetex} \or \boolean{luatex}}{%
%% begin: xelatex and lualatex-specific configuration
\ifxetex\usepackage{xltxtra}\fi
%% realscripts is the only part of xltxtra relevant to lualatex 
\ifluatex\usepackage{realscripts}\fi
%% fontspec package provides extensive control of system fonts,
%% meaning *.otf (OpenType), and apparently *.ttf (TrueType)
%% that live *outside* your TeX/MF tree, and are controlled by your *system*
%% (it is possible that a TeX distribution will place fonts in a system location)
\usepackage{fontspec}
%% We use Latin Modern (lmodern) as the default font
%% So we check that it is available as a system font
\IfFontExistsTF{Latin Modern Roman}{}{\GenericError{}{The font "Latin Modern Roman" requested by PreTeXt output is not available as a system font}{Consult the PreTeXt Guide for help with LaTeX fonts.}{}}
%% We then define various font family commands using a vanilla version,
%% with the intention of letting a style override these choices
%% \setmainfont can be re-issued, and \renewfontfamily can redefine others
\setmainfont{Latin Modern Roman}[SmallCapsFont={Latin Modern Roman Caps}, SlantedFont={Latin Modern Roman Slanted}]
\newfontfamily{\divisionfont}{Latin Modern Roman}
\newfontfamily{\contentsfont}{Latin Modern Roman}
\newfontfamily{\pagefont}{Latin Modern Roman}[SlantedFont={Latin Modern Roman Slanted}]
\newfontfamily{\tabularfont}{Latin Modern Roman}[SmallCapsFont={Latin Modern Roman Caps}]
%% begin: font information supplied by "font-xelatex-style" template
%% end: font information supplied by "font-xelatex-style" template
%% 
%% Extensive support for other languages
\usepackage{polyglossia}
%% Set main/default language based on pretext/@xml:lang value
%% document language code is "en-US", US English
%% usmax variant has extra hypenation
\setmainlanguage[variant=usmax]{english}
%% Enable secondary languages based on discovery of @xml:lang values
%% Enable fonts/scripts based on discovery of @xml:lang values
%% Western languages should be ably covered by Latin Modern Roman
%% end: xelatex and lualatex-specific configuration
}{%
%% begin: pdflatex-specific configuration
\usepackage[utf8]{inputenc}
%% PreTeXt will create a UTF-8 encoded file
%% begin: font setup and configuration for use with pdflatex
%% Portions of a document, are, or may, be affected by font-changing commands
%% These are more robust when using  xelatex  but may be employed with  pdflatex
%% The following definitons are meant to be re-defined in a style with \renewcommand
\newcommand{\divisionfont}{\relax}
\newcommand{\contentsfont}{\relax}
\newcommand{\pagefont}{\relax}
\newcommand{\tabularfont}{\relax}
%% begin: font information supplied by "font-pdflatex-style" template
\usepackage{lmodern}
\usepackage[T1]{fontenc}
%% begin: font information supplied by "font-pdflatex-style" template
%% end: font setup and configuration for use with pdflatex
%% end: pdflatex-specific configuration
}
%% Symbols, align environment, commutative diagrams, bracket-matrix
\usepackage{amsmath}
\usepackage{amscd}
\usepackage{amssymb}
%% allow page breaks within display mathematics anywhere
%% level 4 is maximally permissive
%% this is exactly the opposite of AMSmath package philosophy
%% there are per-display, and per-equation options to control this
%% split, aligned, gathered, and alignedat are not affected
\allowdisplaybreaks[4]
%% allow more columns to a matrix
%% can make this even bigger by overriding with  latex.preamble.late  processing option
\setcounter{MaxMatrixCols}{30}
%%
%%
%% Division Titles, and Page Headers/Footers
%% titlesec package, loading "titleps" package cooperatively
%% See code comments about the necessity and purpose of "explicit" option.
%% The "newparttoc" option causes a consistent entry for parts in the ToC 
%% file, but it is only effective if there is a \titleformat for \part.
%% "pagestyles" loads the  titleps  package cooperatively.
\usepackage[explicit, newparttoc, pagestyles]{titlesec}
%% The companion titletoc package for the ToC.
\usepackage{titletoc}
%% Fixes a bug with transition from chapters to appendices in a "book"
%% See generating XSL code for more details about necessity
\newtitlemark{\chaptertitlename}
%% begin: customizations of page styles via the modal "titleps-style" template
%% Designed to use commands from the LaTeX "titleps" package
%% Plain pages should have the same font for page numbers
\renewpagestyle{plain}{%
\setfoot{}{\pagefont\thepage}{}%
}%
%% Single pages as in default LaTeX
\renewpagestyle{headings}{%
\sethead{\pagefont\slshape\MakeUppercase{\ifthechapter{\chaptertitlename\space\thechapter.\space}{}\chaptertitle}}{}{\pagefont\thepage}%
}%
\pagestyle{headings}
%% end: customizations of page styles via the modal "titleps-style" template
%%
%% Create globally-available macros to be provided for style writers
%% These are redefined for each occurence of each division
\newcommand{\divisionnameptx}{\relax}%
\newcommand{\titleptx}{\relax}%
\newcommand{\subtitleptx}{\relax}%
\newcommand{\shortitleptx}{\relax}%
\newcommand{\authorsptx}{\relax}%
\newcommand{\epigraphptx}{\relax}%
%% Create environments for possible occurences of each division
%% Environment for a PTX "preface" at the level of a LaTeX "chapter"
\NewDocumentEnvironment{preface}{mmmmmm}
{%
\renewcommand{\divisionnameptx}{Preface}%
\renewcommand{\titleptx}{#1}%
\renewcommand{\subtitleptx}{#2}%
\renewcommand{\shortitleptx}{#3}%
\renewcommand{\authorsptx}{#4}%
\renewcommand{\epigraphptx}{#5}%
\chapter*{#1}%
\addcontentsline{toc}{chapter}{#3}
\label{#6}%
}{}%
%% Environment for a PTX "part" at the level of a LaTeX "part"
\NewDocumentEnvironment{partptx}{mmmmmm}
{%
\renewcommand{\divisionnameptx}{Part}%
\renewcommand{\titleptx}{#1}%
\renewcommand{\subtitleptx}{#2}%
\renewcommand{\shortitleptx}{#3}%
\renewcommand{\authorsptx}{#4}%
\renewcommand{\epigraphptx}{#5}%
\part[{#3}]{#1}%
\label{#6}%
}{}%
%% Environment for a PTX "chapter" at the level of a LaTeX "chapter"
\NewDocumentEnvironment{chapterptx}{mmmmmm}
{%
\renewcommand{\divisionnameptx}{Chapter}%
\renewcommand{\titleptx}{#1}%
\renewcommand{\subtitleptx}{#2}%
\renewcommand{\shortitleptx}{#3}%
\renewcommand{\authorsptx}{#4}%
\renewcommand{\epigraphptx}{#5}%
\chapter[{#3}]{#1}%
\label{#6}%
}{}%
%% Environment for a PTX "section" at the level of a LaTeX "section"
\NewDocumentEnvironment{sectionptx}{mmmmmm}
{%
\renewcommand{\divisionnameptx}{Section}%
\renewcommand{\titleptx}{#1}%
\renewcommand{\subtitleptx}{#2}%
\renewcommand{\shortitleptx}{#3}%
\renewcommand{\authorsptx}{#4}%
\renewcommand{\epigraphptx}{#5}%
\section[{#3}]{#1}%
\label{#6}%
}{}%
%%
%% Styles for six traditional LaTeX divisions
\titleformat{\part}[display]
{\divisionfont\Huge\bfseries\centering}{\divisionnameptx\space\thepart}{30pt}{\Huge#1}
[{\Large\centering\authorsptx}]
\titleformat{\chapter}[display]
{\divisionfont\huge\bfseries}{\divisionnameptx\space\thechapter}{20pt}{\Huge#1}
[{\Large\authorsptx}]
\titleformat{name=\chapter,numberless}[display]
{\divisionfont\huge\bfseries}{}{0pt}{#1}
[{\Large\authorsptx}]
\titlespacing*{\chapter}{0pt}{50pt}{40pt}
\titleformat{\section}[hang]
{\divisionfont\Large\bfseries}{\thesection}{1ex}{#1}
[{\large\authorsptx}]
\titleformat{name=\section,numberless}[block]
{\divisionfont\Large\bfseries}{}{0pt}{#1}
[{\large\authorsptx}]
\titlespacing*{\section}{0pt}{3.5ex plus 1ex minus .2ex}{2.3ex plus .2ex}
\titleformat{\subsection}[hang]
{\divisionfont\large\bfseries}{\thesubsection}{1ex}{#1}
[{\normalsize\authorsptx}]
\titleformat{name=\subsection,numberless}[block]
{\divisionfont\large\bfseries}{}{0pt}{#1}
[{\normalsize\authorsptx}]
\titlespacing*{\subsection}{0pt}{3.25ex plus 1ex minus .2ex}{1.5ex plus .2ex}
\titleformat{\subsubsection}[hang]
{\divisionfont\normalsize\bfseries}{\thesubsubsection}{1em}{#1}
[{\small\authorsptx}]
\titleformat{name=\subsubsection,numberless}[block]
{\divisionfont\normalsize\bfseries}{}{0pt}{#1}
[{\normalsize\authorsptx}]
\titlespacing*{\subsubsection}{0pt}{3.25ex plus 1ex minus .2ex}{1.5ex plus .2ex}
\titleformat{\paragraph}[hang]
{\divisionfont\normalsize\bfseries}{\theparagraph}{1em}{#1}
[{\small\authorsptx}]
\titleformat{name=\paragraph,numberless}[block]
{\divisionfont\normalsize\bfseries}{}{0pt}{#1}
[{\normalsize\authorsptx}]
\titlespacing*{\paragraph}{0pt}{3.25ex plus 1ex minus .2ex}{1.5em}
%%
%% Styles for five traditional LaTeX divisions
\titlecontents{part}%
[0pt]{\contentsmargin{0em}\addvspace{1pc}\contentsfont\bfseries}%
{\Large\thecontentslabel\enspace}{\Large}%
{}%
[\addvspace{.5pc}]%
\titlecontents{chapter}%
[0pt]{\contentsmargin{0em}\addvspace{1pc}\contentsfont\bfseries}%
{\large\thecontentslabel\enspace}{\large}%
{\hfill\bfseries\thecontentspage}%
[\addvspace{.5pc}]%
\dottedcontents{section}[3.8em]{\contentsfont}{2.3em}{1pc}%
\dottedcontents{subsection}[6.1em]{\contentsfont}{3.2em}{1pc}%
\dottedcontents{subsubsection}[9.3em]{\contentsfont}{4.3em}{1pc}%
%%
%% Begin: Semantic Macros
%% To preserve meaning in a LaTeX file
%%
%% \mono macro for content of "c", "cd", "tag", etc elements
%% Also used automatically in other constructions
%% Simply an alias for \texttt
%% Always defined, even if there is no need, or if a specific tt font is not loaded
\newcommand{\mono}[1]{\texttt{#1}}
%%
%% Following semantic macros are only defined here if their
%% use is required only in this specific document
%%
%% Used for inline definitions of terms
\newcommand{\terminology}[1]{\textbf{#1}}
%% End: Semantic Macros
%% Division Numbering: Chapters, Sections, Subsections, etc
%% Division numbers may be turned off at some level ("depth")
%% A section *always* has depth 1, contrary to us counting from the document root
%% The latex default is 3.  If a larger number is present here, then
%% removing this command may make some cross-references ambiguous
%% The precursor variable $numbering-maxlevel is checked for consistency in the common XSL file
\setcounter{secnumdepth}{3}
%%
%% AMS "proof" environment is no longer used, but we leave previously
%% implemented \qedhere in place, should the LaTeX be recycled
\newcommand{\qedhere}{\relax}
%%
%% A faux tcolorbox whose only purpose is to provide common numbering
%% facilities for most blocks (possibly not projects, 2D displays)
%% Controlled by  numbering.theorems.level  processing parameter
\newtcolorbox[auto counter, number within=section]{block}{}
%%
%% This document is set to number PROJECT-LIKE on a separate numbering scheme
%% So, a faux tcolorbox whose only purpose is to provide this numbering
%% Controlled by  numbering.projects.level  processing parameter
\newtcolorbox[auto counter, number within=section]{project-distinct}{}
%% A faux tcolorbox whose only purpose is to provide common numbering
%% facilities for 2D displays which are subnumbered as part of a "sidebyside"
\newtcolorbox[auto counter, number within=tcb@cnt@block, number freestyle={\noexpand\thetcb@cnt@block(\noexpand\alph{\tcbcounter})}]{subdisplay}{}
%%
%% tcolorbox, with styles, for EXAMPLE-LIKE
%%
%% example: fairly simple numbered block/structure
\tcbset{ examplestyle/.style={bwminimalstyle, runintitlestyle, blockspacingstyle, after title={\space}, after upper={\space\space\hspace*{\stretch{1}}\(\square\)}, } }
\newtcolorbox[use counter from=block]{example}[2]{title={{Example~\thetcbcounter\notblank{#1}{\space\space#1}{}}}, phantomlabel={#2}, breakable, parbox=false, after={\par}, examplestyle, }
%%
%% tcolorbox, with styles, for PROJECT-LIKE
%%
%% activity: fairly simple numbered block/structure
\tcbset{ activitystyle/.style={bwminimalstyle, runintitlestyle, blockspacingstyle, after title={\space}, } }
\newtcolorbox[use counter from=project-distinct]{activity}[2]{title={{Activity~\thetcbcounter\notblank{#1}{\space\space#1}{}}}, phantomlabel={#2}, breakable, parbox=false, after={\par}, activitystyle, }
%% Localize LaTeX supplied names (possibly none)
\renewcommand*{\partname}{Part}
\renewcommand*{\chaptername}{Chapter}
%% More flexible list management, esp. for references
%% But also for specifying labels (i.e. custom order) on nested lists
\usepackage{enumitem}
%% hyperref driver does not need to be specified, it will be detected
%% Footnote marks in tcolorbox have broken linking under
%% hyperref, so it is necessary to turn off all linking
%% It *must* be given as a package option, not with \hypersetup
\usepackage[hyperfootnotes=false]{hyperref}
%% Hyperlinking active in electronic PDFs, all links solid and blue
\hypersetup{colorlinks=true,linkcolor=blue,citecolor=blue,filecolor=blue,urlcolor=blue}
\hypersetup{pdftitle={Calculus \& Analytical Geometry II}}
%% If you manually remove hyperref, leave in this next command
\providecommand\phantomsection{}
%% If tikz has been loaded, replace ampersand with \amp macro
%% extpfeil package for certain extensible arrows,
%% as also provided by MathJax extension of the same name
%% NB: this package loads mtools, which loads calc, which redefines
%%     \setlength, so it can be removed if it seems to be in the 
%%     way and your math does not use:
%%     
%%     \xtwoheadrightarrow, \xtwoheadleftarrow, \xmapsto, \xlongequal, \xtofrom
%%     
%%     we have had to be extra careful with variable thickness
%%     lines in tables, and so also load this package late
\usepackage{extpfeil}
%% Custom Preamble Entries, late (use latex.preamble.late)
%% Begin: Author-provided packages
%% (From  docinfo/latex-preamble/package  elements)
%% End: Author-provided packages
%% Begin: Author-provided macros
%% (From  docinfo/macros  element)
%% Plus three from MBX for XML characters

\newcommand{\lt}{<}
\newcommand{\gt}{>}
\newcommand{\amp}{&}
%% End: Author-provided macros
\begin{document}
\frontmatter
%% begin: half-title
\thispagestyle{empty}
{\centering
\vspace*{0.28\textheight}
{\Huge Calculus \& Analytical Geometry II}\\}
\clearpage
%% end:   half-title
%% begin: adcard
\thispagestyle{empty}
\null%
\clearpage
%% end:   adcard
%% begin: title page
%% Inspired by Peter Wilson's "titleDB" in "titlepages" CTAN package
\thispagestyle{empty}
{\centering
\vspace*{0.14\textheight}
%% Target for xref to top-level element is ToC
\addtocontents{toc}{\protect\hypertarget{x:book:math142}{}}
{\Huge Calculus \& Analytical Geometry II}\\[3\baselineskip]
{\Large William C. Kronholm}\\[0.5\baselineskip]
{\Large Whittier College}\\}
\clearpage
%% end:   title page
%% begin: copyright-page
\thispagestyle{empty}
\vspace*{\stretch{2}}
\vspace*{\stretch{1}}
\null\clearpage
%% end:   copyright-page
%
%
\typeout{************************************************}
\typeout{Preface  Preface}
\typeout{************************************************}
%
\begin{preface}{Preface}{}{Preface}{}{}{x:preface:preface}
This is a collection of definitions, examples, and activities intended for use in a typical second semester calculus course. The activities are designed to introduce new ideas to students in class as they work in groups.%
\par
The author is in the process of recoding this material from its original LaTeX source code to PreTeXt code. The materials may appear incomplete at times, as the process is ongoing.%
\end{preface}
%% begin: table of contents
%% Adjust Table of Contents
\setcounter{tocdepth}{0}
\renewcommand*\contentsname{Contents}
\tableofcontents
%% end:   table of contents
\mainmatter
%
%
%
\typeout{************************************************}
\typeout{Part I Sequences and Summations}
\typeout{************************************************}
%
\begin{partptx}{Sequences and Summations}{}{Sequences and Summations}{}{}{x:part:part-sequences-and-series}
 To make sense of Riemann sums, we will first study sequences and summations. %
%
\typeout{************************************************}
\typeout{Chapter 1 Sequences}
\typeout{************************************************}
%
\begin{chapterptx}{Sequences}{}{Sequences}{}{}{x:chapter:chapter-sequences}
%
%
\typeout{************************************************}
\typeout{Section 1.1 Introduction}
\typeout{************************************************}
%
\begin{sectionptx}{Introduction}{}{Introduction}{}{}{g:section:idm21}
Intuitively, a sequence is a list of numbers. More formally, a \terminology{sequence} \(\{a_n\}_{n \geq m}\) is the list of numbers \(a_m, a_{m+1}, a_{m+2}, \dots\).%
\begin{example}{A Sequence.}{g:example:idm27}%
The sequence \(\{a_n\}_{n \geq 2}\) where \(a_n = 3n-4\) is the sequnce below:%
\begin{equation*}
2, 5, 8, 11, 14, 17, \dots
\end{equation*}
We might more often describe that sequence by declaring that \(a_n = 3n-4\) for \(n \geq 2\)%
\end{example}
\begin{activity}{}{g:activity:idm35}%
List the first five terms of the sequences below.%
\begin{enumerate}[font=\bfseries,label=(\alph*),ref=\alph*]
\item{}\(a_n = n\) for \(n \geq 0\)\item{}\(b_n = -2n-1\) for \(n \geq 2\)\item{}\(c_n = \frac{2n+1}{n}\) for \(n \geq 1\)\item{}\(d_n = \left(\frac{2}{3}\right)^n\) for \(n \geq 0\)\item{}\(e_n = (-1)^n\) for \(n \geq 0\)\end{enumerate}
\end{activity}
A \terminology{closed form} for a sequence \(a_m, a_{m+1}, \dots\) is an explicit formula for the sequence.%
\begin{example}{Closed Form.}{g:example:idm56}%
A closed form for the sequence \(2, 4, 6, 8, \dots\) is \(a_n = 2n\) for \(n \geq 1\).%
\par
Another closed form for the same sequence is \(a_n = 2(n-1)\) for \(n \geq 0\)%
\end{example}
\begin{activity}{}{g:activity:idm65}%
Determine a closed form expression for the sequences below. That is, write each as \(a_n = \textrm{something}\), \(n \geq m\). (You get to choose \(m\).) There are many (many!) correct expressions.%
\begin{enumerate}[font=\bfseries,label=(\alph*),ref=\alph*]
\item{}\(2, 5, 8, 11, 14, \dots\)\item{}\(-3, 6, -12, 24, -48, \dots\)\item{}\(\frac{2}{3}, \frac{4}{5}, \frac{6}{7}, \frac{8}{9}, \frac{10}{11}, \dots\)\item{}\(-1, 1, -1, 1, -1, \dots\)\item{}\(1, 2, 1, 4, 1, 6, 1, 8, \dots\)\end{enumerate}
\end{activity}
\end{sectionptx}
%
%
\typeout{************************************************}
\typeout{Section 1.2 Convergence}
\typeout{************************************************}
%
\begin{sectionptx}{Convergence}{}{Convergence}{}{}{g:section:idm81}
For us, the most important feature of a sequence will be whether or not it converges, and, in the case of convergence, what it converges to. The definition of convergence should look familiar:%
\par
A sequence \(a_n\) \terminology{converges} to the real number \(L\) if given any \(\varepsilon > 0\) there is an integer \(N\) so that whenever \(n \geq N\), \(|a_n - L| < \varepsilon\). That is, if \(a_n\) is always close to \(L\) when \(n\) is big, then \(a_n\) converges to \(L\). We write this in the usual way: \(\displaystyle\lim_{n \to \infty}a_n = L\). Otherwise, the sequence \terminology{diverges}.%
\par
If \(a_n\) is a sequence and for any \(M > 0\) there is an integer \(N\) so that whenever \(n \geq N\), \(a_n > M\), then we say \(a_n\) \terminology{diverges to infinity} and write \(\displaystyle\lim_{n \to \infty}a_n = \infty\). If \(a_n\) is a sequence and for any \(M < 0\) there is an integer \(N\) so that whenever \(n \geq N\), \(a_n < M\), then we say \(a_n\) \terminology{diverges to negative infinity} and write \(\displaystyle\lim_{n \to \infty}a_n = -\infty\).%
\begin{example}{}{g:example:idm116}%
\(\displaystyle\lim_{n \to \infty} 3n-4 = \infty\) and \(\displaystyle\lim_{n \to \infty} \frac{3n-4}{2n+1} = \frac{3}{2}\).\end{example}
\begin{activity}{}{g:activity:idm119}%
For each of the sequences on the previous page, determine which converge and which do not. In case of convergence, determine the limit. In case of divergence, determine which diverge to infinity, which diverge to negative infinity, and which simply diverge.%
\end{activity}
\begin{activity}{}{g:activity:idm122}%
Compute the limit as \(n\) goes to \(\infty\) of the sequences below. Most of the usual rules for computing limits of functions apply in the context of sequences.%
\begin{enumerate}[font=\bfseries,label=(\alph*),ref=\alph*]
\item{}\(a_n = \frac{1}{n}\), \(n \geq 1\)\item{}\(b_n = \frac{3n-4}{1-4n}\), \(n \geq 0\)\item{}\(c_n = \left(\frac{3}{4}\right)^n\), \(n \geq 0\)\item{}\(d_n = \frac{1}{n} - \frac{1}{n-1}\), \(n \geq 2\)\item{}\(e_n = \sin(\pi n)\), \(n \geq 0\)\end{enumerate}
\end{activity}
\begin{activity}{}{g:activity:idm142}%
Give an example of a sequence \(a_n\) and a continuous, non-zero function \(f\) so that the sequence \(a_n\) diverges, but the sequence \(f(a_n)\) converges. Show that both \(a_n\) and \(f(a_n)\) have the stated properties.%
\end{activity}
\begin{activity}{}{g:activity:idm151}%
Create a sequence \(a_n\) so that \(0 \leq a_n \leq 3\) for all \(n\), and \(a_n\) is \terminology{increasing} in the sense that \(a_n \leq a_{n+1}\) for all \(n\). Show that your sequence has the required properties. Does your sequence converge?%
\end{activity}
\begin{activity}{}{g:activity:idm161}%
Create a sequence \(a_n\) so that \(1 \leq a_n \leq 4\) for all \(n\), and \(a_n\) is \terminology{decreasing} in the sense that \(a_n \geq a_{n+1}\) for all \(n\). Show that your sequence has the required properties. Does your sequence converge?%
\end{activity}
\begin{activity}{}{g:activity:idm171}%
Create a sequence \(a_n\) so that \(2 \leq a_n \leq 3\) for all \(n\), and \(a_n\) diverges. Is \(a_n\) increasing? Decreasing?%
\end{activity}
\begin{activity}{}{g:activity:idm179}%
Is it true that if \(a_n \leq b_n\) for all \(n\), and \(b_n\) is convergent, then \(a_n\) is also convergent? If so, explain why. If not, give an example to show this.%
\end{activity}
\begin{activity}{}{g:activity:idm186}%
Is it true that if \(a_n \geq b_n\) for all \(n\), and \(b_n\) is divergent, then \(a_n\) is also divergent? If so, explain why. If not, give an example to show this.%
\end{activity}
\end{sectionptx}
\end{chapterptx}
 %
%
\typeout{************************************************}
\typeout{Chapter 2 Summations}
\typeout{************************************************}
%
\begin{chapterptx}{Summations}{}{Summations}{}{}{x:chapter:chapter-summations}
%
%
\typeout{************************************************}
\typeout{Section 2.1 Introduction}
\typeout{************************************************}
%
\begin{sectionptx}{Introduction}{}{Introduction}{}{}{g:section:idm195}
Given a sequence \(a_k\), the \terminology{summation} of \(a_k\) as \(k\) goes from \(m\) to \(n\) is denoted%
\begin{equation*}
\sum_{k=m}^n a_k = a_m + a_{m+1} + \cdots + a_n
\end{equation*}
The variable \(k\) used for the \terminology{index of summation} is arbitrary.%
\begin{example}{A Summation.}{g:example:idm207}%
\(\displaystyle\sum_{k = 1}^5 k^2 = 1 + 4 + 9 + 16 + 25\)%
\end{example}
\begin{activity}{}{g:activity:idm211}%
Determine the summations below.%
\begin{enumerate}[font=\bfseries,label=(\alph*),ref=\alph*]
\item{}\(\displaystyle\sum_{k = -2}^5 k\)\item{}\(\displaystyle\sum_{j = 0}^3 2j^2+5j-1\)\item{}\(\displaystyle\sum_{k = 0}^5 2\)\item{}\(\displaystyle\sum_{\ell = -3}^3 \ell^3\)\end{enumerate}
\end{activity}
\begin{activity}{}{g:activity:idm222}%
Combine the summations below into a single summation. Do not compute the sum.%
\begin{enumerate}[font=\bfseries,label=(\alph*),ref=\alph*]
\item{}\(\displaystyle\sum_{k=0}^{25} (k+1)(k-1) + 3\sum_{k=0}^{25} k(k+1)\)\item{}\(\displaystyle\sum_{k=0}^{25} (k+1)(k-1) + 3\sum_{j=0}^{25} j(j+1)\)%
\par
\emph{Hint}: The index of summation is arbitrary.%
\item{}\(\displaystyle\sum_{k=0}^{25} (k+1)(k-1) + 3\sum_{k=2}^{27} k(k+1)\)%
\par
\emph{Hint}: Shift the index of summation so that the two sums start at the same place, but don't change the value of the sum!%
\end{enumerate}
\end{activity}
\end{sectionptx}
%
%
\typeout{************************************************}
\typeout{Section 2.2 Telescoping Sums}
\typeout{************************************************}
%
\begin{sectionptx}{Telescoping Sums}{}{Telescoping Sums}{}{}{g:section:idm237}
Given a function \(f\), the \terminology{finite difference} function \(\Delta f\) is given by \(\Delta f(k) = f(k) - f(k-1)\).%
\begin{example}{A Finite Difference.}{g:example:idm244}%
 If \(f(x) = 2x+1\), then \(\Delta f(3) = f(3) - f(2) = 7-5\)\end{example}
\begin{activity}{}{g:activity:idm248}%
Let \(f(x) = x^2-x+1\). Compute the following.%
\begin{enumerate}[font=\bfseries,label=(\alph*),ref=\alph*]
\item{}\(\Delta f(1)\)\item{}\(\Delta f(2)\)\item{}\(\Delta f(3)\)\item{}\(\Delta f(4)\)\item{}\(\displaystyle\sum_{k=1}^4\Delta f(k)\)\end{enumerate}
\end{activity}
Sometimes we specify the finite difference with simpler notation.%
\begin{example}{Another Way to Write Finite Differences.}{g:example:idm263}%
If \(f(x) = 3x-2\), then we can write \(\Delta f(k)\) as simply \(\Delta (3k-2)\). In this case, \(\Delta (3k-2) = (3k-2) - (3(k-1)-2) = 3k - 2 - (3k - 3 -2) = 3\).%
\end{example}
\begin{activity}{}{g:activity:idm270}%
Compute the following.%
\begin{enumerate}[font=\bfseries,label=(\alph*),ref=\alph*]
\item{}\(\Delta k\)\item{}\(\Delta k^2\)\item{}\(\Delta k^3\)\item{}\(\Delta 1\)\end{enumerate}
\end{activity}
\begin{activity}{}{g:activity:idm281}%
Let \(f\) be a function.%
\begin{enumerate}[font=\bfseries,label=(\alph*),ref=\alph*]
\item{}Expand the summation \(\displaystyle\sum_{k = 1}^3 \Delta f(k)\). Is there any simplification to do? If so, simplify.%
\item{}Without expanding, simplify: \(\displaystyle\sum_{k = 1}^{10} \Delta f(k)\)\item{}Without expanding, simplify: \(\displaystyle\sum_{k = 3}^{10} \Delta f(k)\)\item{}Without expanding, simplify: \(\displaystyle\sum_{k = m+1}^{n} \Delta f(k)\)\end{enumerate}
\end{activity}
Summations of the form \(\displaystyle\sum_{k = m+1}^n \Delta f(k)\) are called \terminology{telescoping sums}, a reference to the collapsible telescopes of yore. These are some of the easiest summations to deal with, since we can determine the value of the sum \emph{without} computing all of the terms of the sequence being summed.%
\end{sectionptx}
%
%
\typeout{************************************************}
\typeout{Section 2.3 Some Useful Summation Formulas}
\typeout{************************************************}
%
\begin{sectionptx}{Some Useful Summation Formulas}{}{Some Useful Summation Formulas}{}{}{g:section:idm298}
\begin{activity}{}{g:activity:idm300}%
It will be convenient for us to have simple formulas for computing the sums \(\displaystyle\sum_{k=1}^n k\), \(\displaystyle\sum_{k=1}^n k^2\), and similar types of sums. Conveniently, we can view these as telescoping sums.%
\begin{enumerate}[font=\bfseries,label=(\alph*),ref=\alph*]
\item{}Show that \(k = \Delta \left[\frac{k(k+1)}{2}\right]\). That is, show that the right-hand side of the equation simplifies to \(k\).%
\item{}Use the previous part to determine a formula for \(\displaystyle\sum_{k=1}^n k\).%
\item{}Show that \(k^2 = \Delta \left[ \frac{k(k+1)(2k+1)}{6}\right]\).%
\item{}Use the previous part to determine a formula for \(\displaystyle\sum_{k=1}^n k^2\).%
\end{enumerate}
\end{activity}
\end{sectionptx}
%
%
\typeout{************************************************}
\typeout{Section 2.4 Geometric Sequences and Sums}
\typeout{************************************************}
%
\begin{sectionptx}{Geometric Sequences and Sums}{}{Geometric Sequences and Sums}{}{}{g:section:idm318}
A sequence of the form \(a_n = b\cdot r^n\) where \(b\) and \(r\) are fixed real numbers is called a \terminology{geometric sequence}.%
\par
A sum of the form \(\displaystyle\sum_{k=m}^n b\cdot r^k\) is called a \terminology{geometric sum}. Notice that a geometric sum is a sum of terms of a geometric sequence.%
\begin{activity}{}{g:activity:idm328}%
Let \(r\) be any real number, but \(r \neq 1\).%
\begin{enumerate}[font=\bfseries,label=(\alph*),ref=\alph*]
\item{}Show that \(\displaystyle r^k = \Delta \left[\frac{r^{k+1}}{r-1}\right]\).%
\item{}Let \(b\) be any real number. Compute \(\displaystyle\sum_{k=0}^n b r^k\).%
\item{}Compute \(\displaystyle\sum_{k=0}^{10} 3 \left(\frac{4}{5}\right)^k\).%
\item{}Compute \(\displaystyle\sum_{k=0}^8  \left(-\frac{5}{4}\right)^k\).%
\end{enumerate}
\end{activity}
\end{sectionptx}
\end{chapterptx}
 %
%
\typeout{************************************************}
\typeout{Chapter 3 Riemann Sums}
\typeout{************************************************}
%
\begin{chapterptx}{Riemann Sums}{}{Riemann Sums}{}{}{x:chapter:chapter-riemann-sums}
%
%
\typeout{************************************************}
\typeout{Section 3.1 Introduction}
\typeout{************************************************}
%
\begin{sectionptx}{Introduction}{}{Introduction}{}{}{g:section:idm348}
Recall that the \terminology{definite integral of \(f\) on the interval \([a,b]\)} is defined to be%
\begin{equation*}
\int_a^b f(x) dx = \lim_{n \to \infty} \sum_{i=1}^n f(x_i)\Delta x
\end{equation*}
where \(\displaystyle \Delta x = \frac{b-a}{n}\) and \(x_i = a + i \Delta x\) for \(i = 1, \dots, n\). This definite integral represents the area under the curve \(f\) from \(a\) to \(b\).%
\begin{activity}{}{g:activity:idm361}%
Let \(f(x) = x+1\) and \([a,b] = [-1,2]\).%
\begin{enumerate}[font=\bfseries,label=(\alph*),ref=\alph*]
\item{}Make a sketch which indicates what \(\displaystyle\int_{-1}^2 f(x) \; dx\) represents.\item{}Determine \(\Delta x\)%
\item{}Determine \(x_i\).%
\item{}Determine the Riemann sum \(\displaystyle\sum_{i=1}^n f(x_i) \Delta x\). Simplify.%
\item{}If you didn't already, make use of the fact that \(\displaystyle \sum_{i=1}^n i = \frac{n(n+1)}{2}\) and \(\displaystyle \sum_{i=1}^n 1 = n\) to further simplify your Riemann sum.%
\item{}Compute \(\displaystyle\lim_{n \to \infty} \sum_{i=1}^n f(x_i) \Delta x\).%
\item{}Compute \(\displaystyle\int_{-1}^2 x+1 \; dx\).%
\item{}Compare your results with your sketch. Does it make sense? Explain.%
\end{enumerate}
\end{activity}
\end{sectionptx}
%
%
\typeout{************************************************}
\typeout{Section 3.2 More Activities}
\typeout{************************************************}
%
\begin{sectionptx}{More Activities}{}{More Activities}{}{}{g:section:idm389}
\begin{activity}{}{g:activity:idm391}%
Repeat the previous activity for the function \(g(x) = x^2\) on the interval \([0,2]\). That is, compute \(\displaystyle \int_0^2 x^2 \; dx\).%
\par
You may need to make use of the fact that \(\displaystyle\sum_{i=1}^n i^2 = \frac{n(n+1)(2n+1)}{6}\).%
\end{activity}
\begin{activity}{}{g:activity:idm399}%
Compute \(\displaystyle\int_{-2}^2 x^2 \;dx\).%
\par
\emph{Hint}: Sketch a graph, look for symmetry, and use a previous result.%
\end{activity}
\begin{activity}{}{g:activity:idm405}%
Let \(f(x) = x^2 -1 \).%
\begin{enumerate}[font=\bfseries,label=(\alph*),ref=\alph*]
\item{}Compute \(\displaystyle \int_{-1}^1 f(x)\; dx\).\item{}Is your result positive or negative? Draw a picture, and use it to explain why this makes sense.%
\end{enumerate}
\end{activity}
\begin{activity}{}{g:activity:idm413}%
Suppose that \(\displaystyle \int_0^3 f(x)\;dx = 0\).%
\begin{enumerate}[font=\bfseries,label=(\alph*),ref=\alph*]
\item{}Explain what this means in terms of the area under \(f\) from \(0\) to \(3\).%
\item{}Draw at least three different graphs of functions where \(\displaystyle \int_0^3 f(x)\;dx = 0\). You do not need to come up with an explicit formula for your functions.%
\end{enumerate}
\end{activity}
\begin{activity}{}{g:activity:idm425}%
Let \(f(x) = 1-3x^2\).%
\begin{enumerate}[font=\bfseries,label=(\alph*),ref=\alph*]
\item{}Find all values of \(b\) so that \(\displaystyle\int_0^b 1-3x^2\; dx = 0\).%
\item{}Sketch a graph to illustrate each case.%
\end{enumerate}
\end{activity}
\end{sectionptx}
\end{chapterptx}
\end{partptx}
%
%
\typeout{************************************************}
\typeout{Part II Integration}
\typeout{************************************************}
%
\begin{partptx}{Integration}{}{Integration}{}{}{x:part:part-integration}
 This worksheet discusses material from section 1.1 of your book. We begin with some friendly review.%
\end{partptx}
\end{document}